\documentclass[13pt, a4paper, twoside]{article}
\usepackage[utf8]{inputenc}
\usepackage{geometry}
\usepackage[czech]{babel}
\usepackage{chemformula}
\usepackage{chemfig}
\usepackage{enumitem}
\usepackage{float}
\usepackage{fancyhdr}
\usepackage{caption}
\usepackage{setspace}
\usepackage{multicol}
\geometry{legalpaper, margin=1.05in}
\pagestyle{fancy}
\lhead{\Large Matěj Červenka, studijní skupina 160}
\rhead{\large 1.12.2023}
\begin{document}
\begin{center}
    \Huge
    Příprava dekahydrátu síranu sodného
\end{center}
\large \onehalfspacing
\section*{Úloha}
Příprava 15g dekahydrátu síranu sodného (Glauberova sůl) rušenou krystalizací ze
získaného roztoku síranu sodného.
\section*{Potřebné chemikálie}
$NaOH$, $H_2SO_4$
\section*{Reakční schéma}
$2NaOH + H2SO_4 \to Na_2SO_4 + 2H_2O$

\section*{Pomůcky}

\section*{Potřebné údaje}

\section*{Postup}
\begin{enumerate}
    \item Vypočítáme potřebná množství $H_2SO_4$  a NaOH pro přípravu 15g Na2SO4·10H2O rušenou krystalizací
    \item Připravíme 10\% roztoky NaOH a $H_2SO_4$
    \item Neutralizace - smícháme oba roztoky, aby měly hodnotu pH 6 až 8. Hodnotu pH zjistíme pomocí indikátorových papírků
    \item Vzniklý roztok $Na_2SO_4$ necháme vykrystalizovat na vodní lázni (na hmotnost asi 28,55g)
    \item Roztok $Na_2SO_4$ necháme vychladnout
    \item Vyloučené krystaly odfiltrujeme (za sníženého tlaku), zvážíme a vypočítáme praktický výtěžek
    
\section*{Výpočty}

\section*{Závěr}


\end{enumerate}

\end{document}